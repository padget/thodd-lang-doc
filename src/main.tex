\NeedsTeXFormat{LaTeX2e}[1995/12/01]
\documentclass{book}
\usepackage[utf8]{inputenc}    
\usepackage[T1]{fontenc}
\usepackage[francais]{babel}   

\title{Thodd : Thought Driven Development}
\date{04/07/2017}


\begin{document}

\maketitle
\tableofcontents

\newpage

\section{Introduction}

Les langages modernes sont pour la plupart de plus en plus simplistes pour permettre au plus grand nombre de commencer la programmation. Mais n'oublions pas que programmer reste avant tout un exercice intellectuel consistant à décrire de la meilleure manière possible ce que doit faire le programme. 

\section{Types simples}

Pour permettre une programmabilité minimale, tout langage se doit de fournir quelques types de base. Thodd n'échappe pas à cette rêgle. On retrouvera donc les nombres, les caractères et les tableaux. Chacun se décline dans plusieurs types de base. 

\subsection{Nombres}

Les types de base étant vus comme nombre sont les entiers positifs, négatifs ainsi que les nombres à virgule flottante. Chacun de ces types sont livrés avec thodd en plusieurs déclinaisons suivant l'interval [min ; max] de valeur qui est nécessaire au sein du programme. Par exemple on ne voudra pas utiliser un type ayant une capacité [$-2^{31}-1$;$+2^{31}$]

\subsubsection{Entiers}

\subsubsection{Flottant}

\subsubsection{Ce qu'il faut retenir}
 


\section{Fonctions}

\section{Plain Of Data (POD)}

\section{Ensembles}

\end{document}