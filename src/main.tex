\NeedsTeXFormat{LaTeX2e}[1995/12/01]
\documentclass{report}
\usepackage[utf8]{inputenc}    
\usepackage[T1]{fontenc}
\usepackage[francais]{babel}   

\title{Thodd : Thought Driven Development}
\date{04/07/2017}


\begin{document}

\maketitle
\tableofcontents

\newpage




\chapter{Introduction}

Les langages modernes sont pour la plupart de plus en 
plus simplistes pour permettre au plus grand nombre de 
commencer la programmation. Mais n'oublions pas que 
programmer reste avant tout un exercice intellectuel 
consistant à décrire de la meilleure manière possible 
ce que doit faire le programme. 





\chapter{Types simples}

Pour permettre une programmabilité minimale, tout langage
se doit de fournir quelques types de base. Thodd n'échappe 
pas à cette rêgle. On retrouvera donc les nombres, les 
caractères et les tableaux. Chacun se décline dans plusieurs 
types de base. 

\section{Nombres}

Les types de base étant vus comme nombre sont les entiers 
positifs, négatifs ainsi que les nombres à virgule flottante. 
Chacun de ces types sont livrés avec thodd en plusieurs 
déclinaisons suivant l'interval [min ; max] de valeur qui 
est nécessaire au sein du programme. Par exemple on ne 
voudra pas utiliser un type ayant une capacité [$-2^{31}-1$;$+2^{31}$]

\subsection{Entiers}

\subsection{Flottant}

\section{Caractères} 

\section{Booléens}

\section{Bytes}

\section{Tableaux}

\section{Ce qu'il faut retenir}






\chapter{Référence, Composant, Valeur}
 
Le langage thodd prend le parti de ne pas avoir de avoir de garbage collector. 
De fait, il faut fournir un système de modélisation permettant de rendre prévisible
la durée de vie des variables dans le programme.

Thodd fournit pour cela trois types de variables :
\begin{itemize}
    \item Valeur
    \item Référence
    \item Composant
\end{itemize} 

\section{Valeur}
Une valeur est le type de variable avec la durée de vie la plus courte et la plus prévisible.
Elle strictement cantonnée au scope dans lequel elle est déclarée. De plus, elle n'est connue qu'une 
fois déclarer. Une déclaration de valeur (ou tout autre chose)  n'est connu du programme qu'après sa 
déclaration.

\begin{verbatim}
    /// x inconnu ici
    x : int = 2
    /// x connu ici
\end{verbatim} 

Lorsque l'on sort du scope de la valeur, celle ci est détruite.

Une valeur est forcement instanciée dans la pile (par opposition au tas).

\section{Référence}

Une référence permet de ...

Une référence est le type de variable avec la durée de vie potentiellement la plus longue 
et la moins prévisible. En réalité, ce n'est pas tout à fait vrai, car on ne peut avoir une 
référence que sur une valeur ou bien un composant. 

Une référence peut être nulle. De telle sorte qu'une référence peut porté durant son existence sur
plusieurs valeurs (mais une seule à un instant t).

\section{Composant}

Un composant permet de mettre en place une relation de dépendance forte entre deux objets. 
L'un sera désigné comme possesseur et l'autre comme possédé. Le cycle de vie du possédé étant
par conséquent lié au possesseur. Un possédé sera au maximum détruit qu'au moment de la destruction
du possesseur.

\begin{verbatim}
    p : person /// Création d'une personne
    a : address /// Création d'une adresse
    a.street = "555 melrose street"
    p.@address = a /// L'adresse a est maintenant sous le contrôle de p via l'attribut address
    /// Ici nous avons utilisé un POD (vu plus tard dans ce manuel). 
\end{verbatim} 

Du point de vue cycle de vie, le composant et la valeur peuvent semblé similaire. La différence
se situe dans le point suivant : une valeur est forcément valorisée alors qu'un composant peut être
null. 
Remarquons la manière dont l'affectation à été faite sur le composant :

\begin{verbatim}
    p.@address = a 
\end{verbatim} 








\chapter{Fonctions}

\section{Fonctions classiques}

\subsection{Déclaration}

\subsection{Utilisation}

\section{Fonctions anonymes : $\lambda$}

@1 + @2 + @0(@1 + @3)



\chapter{Plain Of Data (POD)}

Le POD est la manière centrale de modéliser les structures de données. 





\chapter{Ensembles}

\chapter{Operator sshape}

\chapter{Annotations}

\section{Principe}

\section{Annotation processors}

\end{document}